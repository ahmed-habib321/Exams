\documentclass[]{article}

\input{MyTools}
\usepackage{physics}

\begin{document}
\thispagestyle{empty}
\begin{figure}
    \begin{minipage}{0.7\textwidth}
        \begin{tabular}{l l}
            Department   & : Mathematics and Computer Science \\
            Level        & : Third Level                      \\
            Course Code  & : 040101309                        \\
            Course Title & :  Integral Equations              \\
            Semester     & : Fall 2024                        \\
            Time Allowed & : 30 minutes                       \\
            Lecturer     & : Dr.Hanna R. Ebead                \\
            Total Marks  & : 30 Points                        \\
        \end{tabular}
    \end{minipage}%
    \begin{minipage}{0.3\textwidth}
        \includegraphics[width=4.5cm]{collagelogo.png}
    \end{minipage}
\end{figure}
\vspace*{-1cm}
\begin{center}
    \textbf{\underline{\LARGE Mid-Term Examination Paper}}
\end{center}
\vspace*{.2cm}

\hrule
%{\fontsize{12pt}{12pt}\selectfont

\begin{enumerate}
    \item Solve the integral equations.
          \vspace*{.2cm}
          \begin{enumerate}
              \item $\displaystyle  y(x)=x-\int_{0}^{x} (x-t) y(t)dt $
                    \vspace*{.1cm}
              \item  $\displaystyle y(t)=\frac{7}{8}t+\frac{1}{2} \int_0^1 ts y^2(s) \,ds$
          \end{enumerate}
          \vspace*{.4cm}
    \item

          \begin{enumerate}
              \item Let $\displaystyle f \in C[a,b]$. If $k:[a,b]\times[a,b] \rightarrow \mathbb{R}$ is continuous on $[a,b]\times[a,b]$ Then prove that
                    \[
                        y(t) = f(t) + \lambda \int_a^b k(t,s) y(s) \,ds \quad t\in[a,b]
                    \]
                    has a unique continuous solution $y\in C[a,b]$\\
                    Provided that
                    \(
                    \displaystyle |\lambda|< \frac{1}{L(b-a)} ,\quad \text{Where}\quad L= \max\limits_{t,s\in [a,b]\times[a,b]} |k(t,s)|
                    \)
                    \vspace*{.1cm}
              \item Give an counter example to show that the previous theorem is sufficient but not necessary
          \end{enumerate}
          \vspace*{.4cm}
    \item Convert the following IVP\\
          \[
              \begin{cases}
                  \displaystyle \dv[2]{y}{x} -2x \dv{y}{x} -3y =0 \\\\
                  \displaystyle y(0) =1 , y'(0)=0                 \\
              \end{cases}
          \]
          Into equivalent Volterra integral equation
\end{enumerate}
%}
\vspace*{\fill}
\hrule
\end{document}